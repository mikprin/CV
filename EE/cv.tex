\documentclass{article}
\usepackage{scimisc-cv}
\usepackage{hyperref}
 
\title{ Mikhail Solovyanov CV for Electronic Positions}
\author{Mikhail Solovyanov}
\date{\today}
 
%% These are custom commands defined in scimisc-cv.sty
\cvname{Mikhail Solovyanov, Masters}
\cvpersonalinfo{
Moscow, Russian Federation \cvinfosep
+7-977-905-79-70 \cvinfosep
solovyanov.mm@phystech.edu \cvinfosep
%linkedin
}
 
\begin{document}
 
% \maketitle %% This is LaTeX's default title constructed from \title,\author,\date
 
\makecvtitle %% This is a custom command constructing the CV title from \cvname, \cvpersonalinfo
 
\section{Summary}
\begin{itemize}
\item  Electrical engineer and Interdisciplinary scientist with skills and experience in electronics, computer networks, programming, machine learning and measurements.
\item Led development of a  project resulting in a patent.
\item Self-motivated, problem-solving and collaborative scientist with notable communication skills.
\item Have no stress digging in interdisciplinary fields and learning new subjects on the fly.
\end{itemize}
 
\section{Technical Skills}
 
\begin{itemize}
\item \textbf{Electronics IC design:} Memory design and simulation, digital Electronics simulation, analog and digital IC design, experience simulating and working with ferroelectric capacitors and memristors (cells of ReRAM memory),   mixed signal simulation, \textit{AC,DC,PZ,tran} simulation, parasitic parameters' analysis. Control and signals theory. Operational amplifiers and comparators design. Experience of making layouts for $65 nm$, $90nm$, $180nm$ nodes. 
\item \textbf{Electronics PCB level design:} DC power design, microcontrollers, SoCs, analog electronics, impedance matched design. DIY audio effects design. 
\item \textbf{Programming:} General experience of collaborative functional and object-oriented programming. Working with data and data analysis.
\item \textbf{System administration:} DEVops and Advanced Linux administration including ARM systems, Windows servers, deployment of VPN, SIP and other server client oriented soft.
\item \textbf{Networks and Computers engineering:} Server and PC building for complicated tasks, building custom racks and networks.
\item \textbf{Microscopy/Imaging/Materials:} SEM (Scanning Electron Microscope), Optical Microscope, ion arc and electron evaporator, resist centrifuge, lithography, ellipsometry, semi-professional Photography.
\item \textbf{Mechanical skills:} Soldering (including 0403 SMD), Assembling and general mechanical engineering knowledge and experience.
\item \textbf{Computational and Machine Learning:} Applied machine learning algorithms. General knowledge in framework programming. General knowledge of machine learning methods.
\end{itemize}
 
\section{Software and Hardware Skills}
\begin{itemize}
\item \textbf{Electronics IC design:} Cadence virtuoso, SPICE, SPECTRE.
\item \textbf{Electronics PCB design:} Altium designer, STM32 cube IDE, Arduino IDE, KiCad.
\item \textbf{Programming:} Python, bash, C++, verylog, verylogA, MATLAB.
\end{itemize}
 
\section{Research Experience}
 
%% Another custom command provide by scimisc-cv.sty.
%% First two argumetns are typeset on the first line in bold; 3rd and 4th arguments are typset on second line in italics. 2nd, 3rd and 4th arguments are OPTIONAL
\cvsubsection{\href{https://mipt.ru/english/}{MIPT Neurocomputing systems lab}}[BS + MsC ]
[Engineer ][September 2017 to present]
 
\begin{itemize}
\item This project coordinated by  \href{https://www.scopus.com/authid/detail.uri?authorId=56272708000}{D.Negrov}   led to development of IC's with a $Hf_{0.5} Z_{0.5} O $ based  FRAM with 5nm thin ferroelectric layer.
%\item This project led to 2 conference theses.
\item Last year of work led to completion of essential analog components for memory testing chip. Developed comparators and Op Amps for ADC and DAC of SMU part of a chip.
\item Responsible for development of a prototype of a memory  compiler for new FRAM or ReRAM memory.
\item Used computational methods to evaluate parasitics in prototype IC chips and measuring zonds.
 
\end{itemize}
 
\cvsubsection{ \href{http://twin3d.pro}{Twin3d LLC}}[LLC]
[Leading electrical engineer and Network engineer][January 2021 to present]
 
\begin{itemize}
   \item Built electrical system to trigger and access 150 DSLR cameras in time window of 1ms. This biggest rig in Russia was made for making top edge 3D photorealistic models of people and animals for games, cinema etc.
   \item Developed software for obtaining and sorting photos from 150 cameras with USB, TCP and SSH protocols.
   \item Built full office network for secure storage and access 100TB of data.
   \item Built servers for four RTX 3090 graphics cards for ML load. 
   \end{itemize}
 
\cvsubsection{ \href{http://uvl.io/ }{UVL Robotics LLC}  }[LLC]
[Electronic engineer / System administrator][Feb 2020 to Dec 2020]
 
\begin{itemize}
\item Responsible for development and programming of a motherboard PCB for AI based drone.
\item Created scripts for automated soft building on ARM64 Jetson Xavier NX computer.
\end{itemize}
 
 
%% An example of leaving an argument empty
\cvsubsection{Tech Agent Startup}[][Senior Electronics Engineer][September 2019 to July 2020]
 
\begin{itemize}
\item Developed method to generate electrical impulses read by contact pulse-meter as human pulse.
\item Developed commercial electronic device to work with almost any training apparatus.
\item These projects led to the submission of a patent.
\end{itemize}
 
%% An example of leaving an argument empty
\cvsubsection{\href{https://ailiton.ru/en/}{Ailiton} medical research}[Unimed Group LLC][Senior Electronics Engineer][July 2018 to December 2018]
 
\begin{itemize}
\item Led project focused on the developing a device to read a gel card using machine learning algorithms. Eventually led to the creation of a commercial electronic device.
\end{itemize}
 
 
\section{Education}
 
\begin{itemize}
\item Masters, applied physics and math, Moscow Institute of Physics and Technology (MIPT), 2021
\item BS, Bachelor of applied physics and math, Moscow Institute of Physics and Technology (MIPT), 2019.
\end{itemize}
 
\section{Teaching and Mentoring Experience }
\begin{itemize}
\item 2021 Led Laboratory work (creation of Scotty diode) on the department of solid state physics of MIPT. 
\item 2015-2019 - Mentored 6 undergraduates in their day-to-day physics and math SAT-level exam prep.
\item 2017-2018 - Tutoring in summer camps (foxford.ru)
\end{itemize}
 
\section{Awards}
\begin{itemize}
\item Winner of The 62th MIPT Scientific Conference, in section of nanotechnologies.
\end{itemize}
 
\section{Conference Presentations }
 
\begin{itemize}
\item \href{https://microelectronica.pro/}{International forum microelectronics 2019}. Thesis: "Developing high energy efficient FRAM memory in neurocomputing application".
\item  \href{https://conf62.mipt.ru/}{The 62th MIPT Scientific Conference.}. Thesis "Compiler for high energy efficient FRAM memory in neurocomputing application"
\item \href{https://mipt.ru/science/5top100/education/courseproposal/%D0%A4%D0%AD%D0%A4%D0%9C.pdf}{The 63th MIPT Scientific Conference.}. Thesis: "Development of SMU IC for testing energy efficient memories"
\end{itemize}
 
 
%\section{Publications}
%Still yet to come %
 
 
\section{Other Skills}
\begin{description}[widest=Langauges]
\item[Software] Linux user, Photoshop.
\item[Languages] English: professional proficiency.  Russian: native.
\item[Photography] Have experience in professional photography.
\item[Hobbies] Making DIY audio effects. Competition level dancer (WCS, Hustle), Really love building and maintaining computers.
\end{description}
 
\end{document}