\documentclass{article}
\usepackage{scimisc-cv}
\usepackage{hyperref}
\usepackage{fancyhdr}
\usepackage{graphicx}
\usepackage{xcolor}
\usepackage[export]{adjustbox}
\title{ Mikhail Solovyanov CV for Machine Learning / Data Engineering Positions}
\author{Mikhail Solovyanov}
\date{\today}
 
%% These are custom commands defined in scimisc-cv.sty
\cvname{Mikhail Solovyanov}
\cvpersonalinfo{
27 years old \cvinfosep
Yerevan, Armenia \cvinfosep
+374-44-190-197 \cvinfosep
mikhail.solovyanov@gmail.com \cvinfosep
\href{https://www.linkedin.com/in/mikhail-solovyanov-b4a32b217/}{linkedin} \cvinfosep
\href{https://github.com/mikprin}{github}
}

\begin{document}

\makecvtitle

\section{Machine Learning and Data Science Skills}

\begin{itemize}
   \item \textbf{ML/DL Frameworks:} PyTorch, scikit-learn (sklearn), pandas, numpy, scipy, matplotlib, polars.
   \item \textbf{MLOps and Data Engineering:} AirFlow, Prefect, Celery, FastAPI, SQLAlchemy, Docker, Kubernetes, Helm, Concourse CI/CD, GitHub Actions.
   \item \textbf{Data Handling and Analysis:} SQL, Pandas, NumPy, Polars, ETL pipelines, ML-based model serving, large-scale data processing.
   \item \textbf{High Level Languages:} Python, Bash, HTML, Rust, C.
   \item \textbf{System and Infra Skills:} Linux (CentOS, Debian, Arch), Ansible, Skaffold, Redis, FreeRTOS, Verilog.
\end{itemize}

\section{Work Experience and Research}
\cvsubsection{ \href{https://bostongene.com/}{BostonGene}, Software Engineer (ML-Ops/Data Engineering)}[October 2022 to present]
\begin{itemize}
   \item Developed a Kubernetes-based service for dynamic data query processing from natural language to biomarkers from clinical bioinformatics data, using Python and ML-oriented stacks.
   \item Built and integrated containerized ML-ops services for bioinformatics teams, enabling data scientists to easily deploy and test models in a production environment.
   \item Implemented advanced CI/CD pipelines (Concourse, GitHub Actions) to streamline model deployment, ensuring reproducible machine learning workflows.
   \item Orchestrated services in a dynamic Kubernetes cluster using Helm and Skaffold, ensuring scalable and reliable ML model serving infrastructure.
\end{itemize}

\cvsubsection{ \href{https://en.wikipedia.org/wiki/Synopsys}{Synopsys}, Senior R\&D Engineer}[March 2022 to October 2022]
\begin{itemize}
   \item Devised a prototype flow for testing custom hardware solutions, primarily focusing on data-driven validation and automation with embedded systems.
   \item Presented a "mission mode" testing solution for Qualcomm processors, leveraging data analysis and ML-based evaluation methods for reliability assessments.
\end{itemize}

\cvsubsection{ \href{http://twin3d.pro}{Twin3d}, Lead Engineer (Infrastructure for 3D/Computer Vision Pipelines)}[January 2021 to Feb 2022]
\begin{itemize}
   \item Designed infrastructure to run ML-based 3D reconstruction pipelines, enabling computer vision and photogrammetry tools to scale across multiple servers.
   \item Automated large-scale data handling (240 DSLR cameras) for generating top-edge 3D photorealistic models used in computer vision and VFX pipelines.
   \item Built IoT/cloud solutions integrating Raspberry Pi and ESP32 boards to orchestrate complex data acquisition for ML-driven 3D modeling workflows.
\end{itemize}

\cvsubsection{\href{https://mipt.ru/english/}{MIPT Neurocomputing Systems Lab}, Engineer (ML/Research)[September 2017 to September 2021]}
\begin{itemize}
   \item Developed a memory compiler prototype for emerging FRAM/ReRAM technologies, using Python and Bash, integrating ML methods to optimize memory layouts.
   \item Utilized machine learning to model and evaluate parasitics in prototype IC chips, accelerating the design iteration cycle.
   \item Worked on ML-driven analysis for $Hf_{0.5}Z_{0.5}O$-based FRAM with analog test capabilities, improving energy efficiency metrics in memory design.
\end{itemize}

\cvsubsection{\href{http://uvl.io/ }{UVL Robotics}, Electronic Engineer / DevOps}[Feb 2020 to Dec 2020]
\begin{itemize}
   \item Built and maintained custom Linux distributions with integrated ML frameworks for autonomous, AI-powered quadcopter drones (NVIDIA Jetson Xavier).
   \item Enabled seamless integration of ML object detection and tracking models in embedded systems for autonomous inventory and logistics operations.
\end{itemize}

\section{Education}
\begin{itemize}
\item M.S. in Applied Physics and Math, Moscow Institute of Physics and Technology (MIPT), Dept. of Quantum and Physical Electronics (2019-2021)
\item B.S. in Applied Physics and Math, MIPT, Dept. of Quantum and Physical Electronics (2015-2019)
\end{itemize}

\section{Awards and Achievements}
\begin{itemize}
\item \href{https://bioinf.institute/hack}{Bioinformatics Hackathon'2023}: Task: \href{https://drive.google.com/file/d/16DjU3LqIIUr3Pgvbd6ZjuT_4wTFEsvf1/view?usp=share_link}{BEYOND T2T HUMAN ASSEMBLY}
\item \href{https://microelectronica.pro/}{International forum microelectronics 2019}: Thesis on "Developing high energy efficient FRAM memory in neurocomputing application" involving ML-based optimization.
\item  \href{https://conf62.mipt.ru/}{62nd MIPT Scientific Conference}: Thesis on "Compiler for high energy efficient FRAM memory in neurocomputing application".
\item \href{https://mipt.ru/science/5top100/education/courseproposal/%D0%A4%D0%AD%D0%A4%D0%9C.pdf}{63rd MIPT Scientific Conference}: Thesis on "Development of SMU IC for testing energy efficient memories" with data-driven (ML) analysis.
\item Operate a \href{https://www.youtube.com/channel/UCAjmXQnYQjWoVHx6NIo24CQ}{YouTube Channel} on electronics and software, including ML and data engineering tutorials.
\item Mentored students at MIPT, including those focusing on ML-related research and projects.
\end{itemize}

\section{Other Skills and Interests}
\begin{description}[widest=Languages]
\item[Languages] English: Professional proficiency. Russian: Native.
% \item[Hobbies] Embedded ML in electronics, content creation on YouTube (educational ML and software content), and competitive dancing (WCS, Hustle).
\end{description}
 
\end{document}
