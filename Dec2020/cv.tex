\documentclass{article}
\usepackage{scimisc-cv}

\title{ Mikhail Solovyanov CV for Electronic Positions}
\author{Mikhail Solovyanov}
\date{\today}

%% These are custom commands defined in scimisc-cv.sty
\cvname{Mikhail Solovyanov, Masters}
\cvpersonalinfo{
Moscow, Russian Federation \cvinfosep
+7-977-905-79-70 \cvinfosep
solovyanov.mm@phystech.edu \cvinfosep
%linkedin
}

\begin{document}

% \maketitle %% This is LaTeX's default title constructed from \title,\author,\date

\makecvtitle %% This is a custom command constructing the CV title from \cvname, \cvpersonalinfo

\section{Summary}
\begin{itemize}
\item Interdisciplinary scientist in electronics and electrical engineer  with skills and experience in electronics, programming, machine learning and measurment.

\item Led development of a  project resulting in a patent.
\item Self-motivated, problem-solving and collaborative scientist with notable communication skills.
\item Have no stress digging in interdisciplinary fields and learning new subjects on the fly.
\item Participated in collaborative projects, resulting in publications, including high impact publications.
\end{itemize}

\section{Technical Skills}

\begin{itemize}
\item \textbf{Electronics IC design:} Memory design and simulation, Digital Electronics simulation, analog and digital IC design, Mixed signal simulation, \textit{AC,DC,PZ,tran} simulation , parasitic parameters analysis,
\item \textbf{Electronics PCB level design:} PCB design, microcontroller programming.
\item \textbf{Microscopy/Imaging:} SEM (Scanning Electron Microscope) , Optical Microscope, Ellipsometry, semi-professional Photography.
\item \textbf{Computational and Machine Learning:} Experience in cross platform inux system administration. Have experience in applying machine learning algorithms. General knowledge in framework programming. General knowledge of machine learning methods,
\end{itemize}

\section{Software and Hardware Skills}
\begin{itemize}
\item \textbf{Electronics IC design:} Cadence virtuoso, SPICE, SPECTRE,
\item \textbf{Electronics PCB design:} Altium design, STM32 coding, Arduino coding, KiCad.
\item \textbf{Programming:} Python,verylog, verylogA, MATLAB, Shell script
\end{itemize}

\section{Research Experience}

%% Another custom command provide by scimisc-cv.sty.
%% First two argumetns are typeset on the first line in bold; 3rd and 4th arguments are typset on second line in italics. 2nd, 3rd and 4th arguments are OPTIONAL
\cvsubsection{Neurocomputing systems lab}[MIPT]
[Engineer ][Yanuary 2018 to present]

\begin{itemize}
\item This project coordinated by D.Negrov led to development of IC's with a new type of FRAM with 5nm thin ferroelectric layer.
\item First russian Neural Processing Unit (NPU) IC developed also developed by MIPT NCS lab, use this new FRAM memory.
\item Responsible for development of a memory compiler for new FRAM. Used computational methods to evaluate parasitics in prototype IC chips and measuring zonds.
\item This project led to 2 conference theses.
\end{itemize}


\cvsubsection{UVL Robotics}[]
[Electronic engineer / System administrator][Feb 2020 to present]

\begin{itemize}
\item Responsible for development and programming of a PCB for AI based drone.
\item Created scripts for automated soft building on ARM64 Jetson Xsavier NX computer.
\end{itemize}

%% An example of leaving an argument empty
\cvsubsection{Tech Agent Startup}[][Seniour Electronics Engineer][September 2019 to July 2020]

\begin{itemize}
\item Developed methode to generate electrical impulses read by contact pulsometer as human pulse.
\item Developed commercial electronic device to work with almost any training apparatus.
\item These projects led to the submission of 2 patents.
\end{itemize}

%% An example of leaving an argument empty
\cvsubsection{Ailiton medical recearch}[Unimed Group][Seniour Electronics Engineer][July 2018 to December 2018]

\begin{itemize}
\item Led project focused on the developing a device to read a gel card using maching learning algorithms. Eventually led to the creation of a commercial electronic device.
\end{itemize}


\section{Education}

\begin{itemize}
\item Masters, applied physics and math, Moscow Institute of Physics and Technology (MIPT), 2021
\item BS, Bachelor of applied physics and math , Moscow Institute of Physics and Technology (MIPT), 2019.
\end{itemize}

\section{Teaching and Mentoring Experience }
\begin{itemize}
\item 2015-2019 - Mentored 6 undergraduates in their day-to-day phisics and math SAT-level exam prep.
\item 2017-2018 - Tutoring in summer camps (foxford.ru)
\end{itemize}

\section{Awards}
\begin{itemize}
\item Winner of 62 MIPT conference, in section of nanotechnologies.
\end{itemize}

\section{Conference Presentations }

\begin{itemize}
\item International forum microelectronics 2019. Thesis: "Developing high energy eficient FRAM memory in neurocomputing application".
\item 62 MIPT conference. Thesis "Compiler for high energy eficient FRAM memory in neurocomputing application"
\end{itemize}


\section{Publications}
Still yet to come %


\section{Other Skills}
\begin{description}[widest=Langauges]
\item[Software]	Linux user, Photoshop.
\item[Languages]	English: professional proficiency.  Russian: native.
\item[Photography] Have experience in professional photography.
\end{description}

\end{document}
