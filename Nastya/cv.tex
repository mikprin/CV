\documentclass{article}
\usepackage{scimisc-cv}
\usepackage{hyperref}

\title{ Anastasiia Fokina CV}
\author{Anastasiia Fokina}
\date{\today}

%% These are custom commands defined in scimisc-cv.sty
\cvname{Anastasiia Fokina}
\cvpersonalinfo{
Schaffhausen,Switzerland \cvinfosep
+41-76-264-84-53 \cvinfosep
fokina.as@phystech.edu \cvinfosep
%linkedin
}

\begin{document}

% \maketitle %% This is LaTeX's default title constructed from \title,\author,\date

\makecvtitle %% This is a custom command constructing the CV title from \cvname, \cvpersonalinfo

\section{Summary}
\begin{itemize}
\item Interdisciplinary software engineer with skills and experience in machine learning, physics, and electronics (robotics).
\item Self-motivated, problem-solving and collaborative scientist with notable communication and leadership skills.
\item No lack in collaboration with colleges during work in large projects.
\item Vast interests including technical fields, soft skills,  and art.
\end{itemize}

\section{Technical Skills}

\begin{itemize}
\item \textbf{Software Engineering:} General knowledges in Machine Learning methods. General knowledges in OOP.
\item \textbf{Electronics:} Soldering, microprocessors programming.
\item \textbf{Science:} Wide range of general knowledges in Natural Sciences - \textit{High Mathematics, Physics, Statistics, Electronics }.
\item \textbf{Production:} Photography - \textit{portraits, reportage, food photography, nature , Videography}
\item \textbf{Design:} Photo retouch, 3D-modeling.


\end{itemize}

\section{Software and Hardware Skills}
\begin{itemize}
\item \textbf{Programming:} Python, C, C++, GIT, Arduino coding
\item \textbf{Design Software:} Adobe Photoshop, Adobe Premiere, Adobe Lightroom
\item \textbf{3D Design Software:} Blender, SolidWorks
\item \textbf{Plotting:} Origin, matplotlib
\end{itemize}

\section{Jobs}

%% Another custom command provide by scimisc-cv.sty.
%% First two argumetns are typeset on the first line in bold; 3rd and 4th arguments are typset on second line in italics. 2nd, 3rd and 4th arguments are OPTIONAL
\cvsubsection{Center of Molecular Electronics}[MIPT]
[technician ][September 2019 to July 2020]

\begin{itemize}
\item Providing researches and molecular electronics.
\item Participation in sensitive sensors and geophones development.
\item Project was lead by I. Zaytsev.
\item Conferences participation.
\end{itemize}

\cvsubsection{Media – center of Evening School of Physics and Technology}[MIPT]
[Videographer][2017 to December 2019]

\begin{itemize}
\item Lectures recording.
\item Organization of live broadcasts.
\item Video editing.
\end{itemize}

\cvsubsection{MIPT Media Center/ MediaClub}[MIPT]
[Photograph, photojournalist ][2018 to December 2019]

\begin{itemize}
\item Video recording.
\item Photo reports of MIPT events.
\item Photo retouch.
\end{itemize}

\cvsubsection{Evening School of Physics and Technology }[MIPT]
[Computer Science Teacher ][2018 to December 2019]

\begin{itemize}
\item Video recording.
\item Photo reports of MIPT events \textit{Events: Days of Physics, Igromir 2018, Moscow Discofox Championship, MIPT graduations, Welcome Days e.t.c.}.
\item Photo retouch.
\end{itemize}



\section{Education}

\begin{itemize}
\item Master in Computer Science and Software Engineering , Schaffhausen Institute of Technology, 2020 to present
\item Masters in applied physics and math, Moscow Institute of Physics and Technology (MIPT), 2020 to present
\item BS, Bachelor of applied physics and math , Moscow Institute of Physics and Technology (MIPT), 2020.
\end{itemize}


\section{Awards}
\begin{itemize}
\item 2nd place of 62 MIPT conference, in section of arctic technologies.
\end{itemize}

\section{Conference Papers }

\begin{itemize}
\item A SIMPLE AND CHEAP METHOD OF MET GEOPHONES AND SEISMIC ACCELEROMETERS TEMPERATURE SENSITIVITY STABILIZATION IN A WIDE TEMPERATURE BAND \\
September 2020 \\
DOI: 10.5593/sgem2020/1.2/s05.053 \\
Conference: 20th International Multidisciplinary Scientific GeoConference Proceedings SGEM 2020
\item  METHOD FOR ENSURING TEMPERATURE AND TIME STABILITY OF PARAMETERS OF MOLECULAR ELECTRONIC GEOPHYSICAL SENSORS FOR OIL AND GAS EXPLORATION SYSTEMS\\
September 2020 \\
DOI: 10.5593/sgem2020/1.2/s06.091 \\
Conference: 20th International Multidisciplinary Scientific GeoConference Proceedings SGEM 2020
\end{itemize}


\section{Other Skills}
\begin{description}[widest=Langauges]
\item[Software]	Linux user, LateX
\item[Languages]	English: B2. German A1. Russian: native.

\end{description}

\end{document}
